\documentclass[draft]{article}

% - Style
\usepackage{base}

% - Title
\title{PHYS4000 - Workshop 2}
\author{Tom Ross - 1834 2884}
\date{}

% - Headers
\pagestyle{fancy}
\fancyhf{}
\rhead{\theauthor}
\chead{}
\lhead{\thetitle}
\rfoot{\thepage}
\cfoot{}
\lfoot{}

% - Document
\begin{document}

\section*{Theory}

\subsection*{Basis}

We utilise a complete basis of the form,
$\mathcal{B} = \lrset{\ket{\phi_{i}}}_{i = 1}^{N}$, where the basis functions
are represented in coordinate-space in the form
\begin{equation}
  \label{eq:basis}
  \phi_{i}\lr{r, \Omega}
  =
  \dfrac{1}{r}
  \varphi_{k_{i}, \ell_{i}}\lr{r}
  Y_{\ell_{i}}^{m_{i}}\lr{\Omega}
  \qq{for}
  i = 1, \dotsc, N
\end{equation}
where the radial functions,
$\mathcal{R} = \lrset{\ket{\varphi_{k_{i}, \ell_{i}}}}_{i = 1}^{N}$ form a
complete basis for the radial function space, in the limit as $N \to \infty$.
We utilise a Laguerre basis for the set of radial functions which,
for $k = 1, 2, \dotsc$ and where $\ell \in \lrset{0, 1, \dotsc}$, are of the
following form in coordinate-space
\begin{equation}
  \label{eq:basis-lag}
  \varphi_{k, \ell}\lr{r}
  =
  N_{k, \ell}
  \lr
  {
    2
    \alpha_{\ell}
    r
  }^{\ell + 1}
  \exponential\lr
  {
    -
    \alpha_{\ell}
    r
  }
  L_{k - 1}^{2 \ell + 1}\lr{2 \alpha_{\ell} r}
\end{equation}
where $\alpha_{\ell} \in (0, \infty)$ are arbitrarily chosen constants,
where $N_{k, \ell}$ are the normalisation constants, given by
\begin{equation}
  \label{eq:basis-lag-norm}
  N_{k, \ell}
  =
  \sqrt
  {
    \dfrac
    {
      \alpha_{\ell}
      \lr{k - 1}!
    }
    {
      \lr{k + \ell}
      \lr{k + 2 \ell}!
    }
  }
\end{equation}
and where $L_{k - 1}^{2 \ell + 1}$ are the generalised Laguerre polynomials.

\subsection*{Overlap Matrix Elements}

The overlap matrix elements, $B_{i, j}$, are of the form
\begin{equation}
  \label{eq:me-overlap}
  B_{i, j}
  =
  \bra*{\phi_{i}}
  \ket*{\phi_{j}}
  =
  \bra*{\tfrac{1}{r}\varphi_{k_{i}, \ell_{i}}}
  \ket*{\tfrac{1}{r}\varphi_{k_{j}, \ell_{j}}}
  \bra*{Y_{l_{i}}^{m_{i}}}
  \ket*{Y_{l_{j}}^{m_{j}}}
  =
  \bra*{\tfrac{1}{r}\varphi_{k_{i}, \ell_{i}}}
  \ket*{\tfrac{1}{r}\varphi_{k_{j}, \ell_{j}}}
  \delta_{l_{i}, l_{j}}
  \delta_{m_{i}, m_{j}}
\end{equation}
where
\begin{equation}
  \label{eq:me-overlap-radial}
  \bra*{\tfrac{1}{r}\varphi_{k_{i}, \ell}}
  \ket*{\tfrac{1}{r}\varphi_{k_{j}, \ell}}
  =
  \begin{cases}
    1
    ,
    &
    \qq{if} k_{i} = k_{j}
    \\
    -
    \tfrac{1}{2}
    \sqrt
    {
      1
      -
      \dfrac
      {
        \ell
        \lr{\ell + 1}
      }
      {
        \lr{k_{i} + \ell}
        \lr{k_{i} + \ell + 1}
      }
    }
    ,
    &
    \qq{if}
    k_{j} = k_{i} + 1
    \\
    \bra*{\tfrac{1}{r}\varphi_{k_{j}, \ell}}
    \ket*{\tfrac{1}{r}\varphi_{k_{i}, \ell}}
    ,
    &
    \qq{if}
    k_{i} = k_{j} + 1
    \\
    0
    ,
    &
    \qq{otherwise}
    \\
  \end{cases}
  .
\end{equation}

\subsection*{Kinetic Matrix Elements}

The kinetic matrix elements, $K_{i, j}$, are of the form
\begin{equation}
  \label{eq:me-kinetic}
  K_{i, j}
  =
  \bra*{\phi_{i}}
  \hat{K}
  \ket*{\phi_{j}}
  =
  \alpha^{2}
  \lr
  {
    \delta_{k_{i}, k_{j}}
    -
    \tfrac{1}{2}
    \bra*{\tfrac{1}{r}\varphi_{k_{i}, \ell_{i}}}
    \ket*{\tfrac{1}{r}\varphi_{k_{j}, \ell_{j}}}
  }
  \delta_{l_{i}, l_{j}}
  \delta_{m_{i}, m_{j}}
  .
\end{equation}

\subsection*{Spherically-Symmetric Potential Matrix Elements}
For a spherically symmetric potential, $V\lr{r, \Omega} = V\lr{r}$, the
potential matrix elements can be calculated numerically to be of the form
\begin{equation}
  \label{eq:me-potential-spherical}
  V_{i, j}
  =
  \bra*{\phi_{i}}
  \hat{V}
  \ket*{\phi_{j}}
  =
  \bra*{\tfrac{1}{r}\varphi_{k_{i}, \ell_{i}}}
  \hat{V}
  \ket*{\tfrac{1}{r}\varphi_{k_{j}, \ell_{j}}}
  \delta_{l_{i}, l_{j}}
  \delta_{m_{i}, m_{j}}
\end{equation}
where
\begin{equation}
  \label{eq:me-potential-spherical-radial}
  \bra*{\tfrac{1}{r}\varphi_{k_{i}, \ell_{i}}}
  \hat{V}
  \ket*{\tfrac{1}{r}\varphi_{k_{j}, \ell_{j}}}
  =
  \int_{0}^{\infty}
  \dd{r}
  \varphi_{k_{i}, \ell_{i}}\lr{r}
  V\lr{r}
  \varphi_{k_{j}, \ell_{j}}\lr{r}
  .
\end{equation}

\clearpage

\section{$\mathrm{H}_{2}^{+}$ Potential-Energy Curves}

\subsection*{Details of Relevant Theory and Code}

\subsubsection*{Axially-Symmetric Potential}
The axially-symmetric potential of the $\mathrm{H}_{2}^{+}$ molecule, with two
nuclei at $\vb{R} = \lr{0, 0, \pm \tfrac{R}{2}}$, can be written in the form
\begin{equation}
  \label{eq:potential-axial}
  V\lr{r, \Omega}
  =
  -
  \lr[\bigg]
  {
    \dfrac
    {
      1
    }
    {
      \norm
      {
        \vb{r}
        +
        \vb{R}
      }
    }
    +
    \dfrac
    {
      1
    }
    {
      \norm
      {
        \vb{r}
        -
        \vb{R}
      }
    }
  }
\end{equation}
which can be written alternatively, using the multipole expansion, in the form
\begin{equation}
  \label{eq:potential-axial-multipole}
  V\lr{r, \Omega}
  =
  -
  2
  \sum_{\lambda \in E}
  \sqrt
  {
    \dfrac
    {
      4\pi
    }
    {
      2\lambda
      +
      1
    }
  }
  \dfrac
  {
    r_{<}^{\lambda}
    % \lr[\big]{\min\lr{r, \tfrac{R}{2}}}^{\lambda}
  }
  {
    r_{>}^{\lambda + 1}
    % \lr[\big]{\max\lr{r, \tfrac{R}{2}}}^{\lambda + 1}
  }
  Y_{\lambda}^{0}\lr{\Omega}
\end{equation}
where $r_{<} = \min\lr{r, \tfrac{R}{2}}$, $r_{>} = \max\lr{r, \tfrac{R}{2}}$,
and where $E = \lrset{0, 2, 4, \dotsc}$ is the set of even integers.
Note that in computational implementations, we truncate this sum at some term,
$\lambda_{\max}$.
Whence, it follows that the matrix elements for this potential can be calculated
numerically to be of the form
\begin{alignat}{2}
  \label{eq:me-potential-axial}
  V_{i, j}
  =
  \bra*{\phi_{i}}
  \hat{V}
  \ket*{\phi_{j}}
  {}={}
  &
  -
  2
  \sum_{\lambda \in E}^{\lambda_{\max}}
  \lr[\bigg]
  {
    \int_{0}^{\infty}
    \dd{r}
    \varphi_{k_{i}, \ell_{i}}\lr{r}
    \dfrac
    {
      r_{<}^{\lambda}
    }
    {
      r_{>}^{\lambda + 1}
    }
    \varphi_{k_{j}, \ell_{j}}\lr{r}
  }
  \lr[\bigg]
  {
    \sqrt
    {
      \dfrac
      {
        4\pi
      }
      {
        2\lambda
        +
        1
      }
    }
    \bra*{Y_{\ell_{i}}^{m_{i}}}
    Y_{\lambda}^{0}\lr{\Omega}
    \ket*{Y_{\ell_{j}}^{m_{j}}}
  }
  \\
  &
  {}\times{}
  \delta_{\pi_{i}, \pi_{j}}
  \delta_{m_{i}, m_{j}}
  \nonumber
\end{alignat}
where $\pi_{i} = \lr{-1}^{\ell_{i}}$ is the parity quantum number.

\subsubsection*{Basis Symmetry}

Due to the axial symmetry of the $\mathrm{H}_{2}^{+}$ potential, and thus the
Hamiltonian for the electron in this molecule, we may choose a symmetrised
basis $\mathcal{B}^{\lr{m, \pi}}$ with specified azimuthal angular momentum,
$m$, and parity, $\pi$.
Furthermore, due to the computational constraints, we restrict our basis to
having $\lrset{N_{\ell}}_{\ell = 0}^{\ell_{\max}}$ radial functions per $\ell$,
with exponential falloffs $\lrset{\alpha_{\ell}}_{\ell = 0}^{\ell_{\max}}$.
For simplicity, we select $N_{\ell} = N_{0}$, and $\alpha_{\ell} = \alpha_{0}$,
for each $\ell = 0, \dotsc, \ell_{max}$.

\subsection{Comparison with Accurate Potential-Energy Curve for $1s\sigma_{g}$.}

\subsection{Comparison with Accurate Potential-Energy Curve for $2p\sigma_{u}$.}

\section{$\mathrm{H}_{2}^{+}$ Vibrational Wave Functions}

\subsection*{Details of Relevant Theory and Code}

\subsection{Vibrational Wave Functions for $1s\sigma_{g}$ PEC.}

\subsection{Lowest-Energy, $\nu = 0$, Vibrational Wave Functions for
  $1s\sigma_{g}$ PEC, for each Isotopologue of $\mathrm{H}_{2}^{+}$.}

\end{document}