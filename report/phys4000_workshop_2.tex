\documentclass[draft]{article}

% - Style
\usepackage{base}

% - Plotting
%%% TeX-command-extra-options: "-shell-escape"
\usepackage{pgfplotstable}
\usepgfplotslibrary{units}

% - Listings
\usepackage{color}
\usepackage{listings}

\lstset{
  basicstyle=\ttfamily\footnotesize\color{black}
  , commentstyle=\color{blue}
  , keywordstyle=\color{purple}
  , stringstyle=\color{orange}
  %
  , numbers=left
  , numbersep=5pt
  , stepnumber=1
  , numberstyle=\ttfamily\small\color{black}
  %
  , keepspaces=true
  , showspaces=false
  , showstringspaces=false
  , showtabs=false
  , tabsize=2
  , breaklines=true
  %
  , frame=single
  , backgroundcolor=\color{white}
  , rulecolor=\color{black}
  , captionpos=b
}

% file or folder
\lstdefinestyle{ff}{
  basicstyle=\ttfamily\normalsize\color{purple}
}

\newcommand{\lilf}[1]{\lstinline[style=ff]{#1}}

% - Title
\title{PHYS4000 - Workshop 2}
\author{Tom Ross - 1834 2884}
\date{}

% - Headers
\pagestyle{fancy}
\fancyhf{}
\rhead{\theauthor}
\chead{}
\lhead{\thetitle}
\rfoot{\thepage}
\cfoot{}
\lfoot{}

% - Document
\begin{document}

The entire code repository used to calculate the data for this report, and the
\lilf{tex} file used to produce this \lilf{pdf} document, can be found at
\url{https://github.com/dgsaf/acqm-workshop-2}.

\section*{Theory}

\subsection*{Basis}

We utilise a complete basis of the form,
$\mathcal{B} = \lrset{\ket{\phi_{i}}}_{i = 1}^{N}$, where the basis functions
are represented in coordinate-space in the form
\begin{equation}
  \label{eq:basis}
  \phi_{i}\lr{r, \Omega}
  =
  \dfrac{1}{r}
  \varphi_{k_{i}, \ell_{i}}\lr{r}
  Y_{\ell_{i}}^{m_{i}}\lr{\Omega}
  \qq{for}
  i = 1, \dotsc, N
\end{equation}
where the radial functions,
$\mathcal{R} = \lrset{\ket{\varphi_{k_{i}, \ell_{i}}}}_{i = 1}^{N}$ form a
complete basis for the radial function space, in the limit as $N \to \infty$.
We utilise a Laguerre basis for the set of radial functions which,
for $k = 1, 2, \dotsc$ and where $\ell \in \lrset{0, 1, \dotsc}$, are of the
following form in coordinate-space
\begin{equation}
  \label{eq:basis-lag}
  \varphi_{k, \ell}\lr{r}
  =
  N_{k, \ell}
  \lr
  {
    2
    \alpha_{\ell}
    r
  }^{\ell + 1}
  \exponential\lr
  {
    -
    \alpha_{\ell}
    r
  }
  L_{k - 1}^{2 \ell + 1}\lr{2 \alpha_{\ell} r}
\end{equation}
where $\alpha_{\ell} \in (0, \infty)$ are arbitrarily chosen constants,
where $N_{k, \ell}$ are the normalisation constants, given by
\begin{equation}
  \label{eq:basis-lag-norm}
  N_{k, \ell}
  =
  \sqrt
  {
    \dfrac
    {
      \alpha_{\ell}
      \lr{k - 1}!
    }
    {
      \lr{k + \ell}
      \lr{k + 2 \ell}!
    }
  }
\end{equation}
and where $L_{k - 1}^{2 \ell + 1}$ are the generalised Laguerre polynomials.

\subsection*{Overlap Matrix Elements}

The overlap matrix elements, $B_{i, j}$, are of the form
\begin{equation}
  \label{eq:me-overlap}
  B_{i, j}
  =
  \bra*{\phi_{i}}
  \ket*{\phi_{j}}
  =
  \bra*{\tfrac{1}{r}\varphi_{k_{i}, \ell_{i}}}
  \ket*{\tfrac{1}{r}\varphi_{k_{j}, \ell_{j}}}
  \bra*{Y_{l_{i}}^{m_{i}}}
  \ket*{Y_{l_{j}}^{m_{j}}}
  =
  \bra*{\tfrac{1}{r}\varphi_{k_{i}, \ell_{i}}}
  \ket*{\tfrac{1}{r}\varphi_{k_{j}, \ell_{j}}}
  \delta_{l_{i}, l_{j}}
  \delta_{m_{i}, m_{j}}
\end{equation}
where
\begin{equation}
  \label{eq:me-overlap-radial}
  \bra*{\tfrac{1}{r}\varphi_{k_{i}, \ell}}
  \ket*{\tfrac{1}{r}\varphi_{k_{j}, \ell}}
  =
  \begin{cases}
    1
    ,
    &
    \qq{if} k_{i} = k_{j}
    \\
    -
    \tfrac{1}{2}
    \sqrt
    {
      1
      -
      \dfrac
      {
        \ell
        \lr{\ell + 1}
      }
      {
        \lr{k_{i} + \ell}
        \lr{k_{i} + \ell + 1}
      }
    }
    ,
    &
    \qq{if}
    k_{j} = k_{i} + 1
    \\
    \bra*{\tfrac{1}{r}\varphi_{k_{j}, \ell}}
    \ket*{\tfrac{1}{r}\varphi_{k_{i}, \ell}}
    ,
    &
    \qq{if}
    k_{i} = k_{j} + 1
    \\
    0
    ,
    &
    \qq{otherwise}
    \\
  \end{cases}
  .
\end{equation}

\subsection*{Kinetic Matrix Elements}

The kinetic matrix elements, $K_{i, j}$, are of the form
\begin{equation}
  \label{eq:me-kinetic}
  K_{i, j}
  =
  \bra*{\phi_{i}}
  \hat{K}
  \ket*{\phi_{j}}
  =
  \alpha^{2}
  \lr
  {
    \delta_{k_{i}, k_{j}}
    -
    \tfrac{1}{2}
    \bra*{\tfrac{1}{r}\varphi_{k_{i}, \ell_{i}}}
    \ket*{\tfrac{1}{r}\varphi_{k_{j}, \ell_{j}}}
  }
  \delta_{l_{i}, l_{j}}
  \delta_{m_{i}, m_{j}}
  .
\end{equation}

\subsection*{Spherically-Symmetric Potential Matrix Elements}
For a spherically symmetric potential, $V\lr{r, \Omega} = V\lr{r}$, the
potential matrix elements can be calculated numerically to be of the form
\begin{equation}
  \label{eq:me-potential-spherical}
  V_{i, j}
  =
  \bra*{\phi_{i}}
  \hat{V}
  \ket*{\phi_{j}}
  =
  \bra*{\tfrac{1}{r}\varphi_{k_{i}, \ell_{i}}}
  \hat{V}
  \ket*{\tfrac{1}{r}\varphi_{k_{j}, \ell_{j}}}
  \delta_{l_{i}, l_{j}}
  \delta_{m_{i}, m_{j}}
\end{equation}
where
\begin{equation}
  \label{eq:me-potential-spherical-radial}
  \bra*{\tfrac{1}{r}\varphi_{k_{i}, \ell_{i}}}
  \hat{V}
  \ket*{\tfrac{1}{r}\varphi_{k_{j}, \ell_{j}}}
  =
  \int_{0}^{\infty}
  \dd{r}
  \varphi_{k_{i}, \ell_{i}}\lr{r}
  V\lr{r}
  \varphi_{k_{j}, \ell_{j}}\lr{r}
  .
\end{equation}

\clearpage

\section{$\mathrm{H}_{2}^{+}$ Potential-Energy Curves}

\subsection*{Details of Relevant Theory and Code}

\subsubsection*{Axially-Symmetric Potential}
The axially-symmetric potential of the $\mathrm{H}_{2}^{+}$ molecule, with two
nuclei at $\vb{R} = \lr{0, 0, \pm \tfrac{R}{2}}$, can be written in the form
\begin{equation}
  \label{eq:potential-axial}
  V\lr{r, \Omega}
  =
  -
  \lr[\bigg]
  {
    \dfrac
    {
      1
    }
    {
      \norm
      {
        \vb{r}
        +
        \vb{R}
      }
    }
    +
    \dfrac
    {
      1
    }
    {
      \norm
      {
        \vb{r}
        -
        \vb{R}
      }
    }
  }
\end{equation}
which can be written alternatively, using the multipole expansion, in the form
\begin{equation}
  \label{eq:potential-axial-multipole}
  V\lr{r, \Omega}
  =
  -
  2
  \sum_{\lambda \in E}
  \sqrt
  {
    \dfrac
    {
      4\pi
    }
    {
      2\lambda
      +
      1
    }
  }
  \dfrac
  {
    r_{<}^{\lambda}
    % \lr[\big]{\min\lr{r, \tfrac{R}{2}}}^{\lambda}
  }
  {
    r_{>}^{\lambda + 1}
    % \lr[\big]{\max\lr{r, \tfrac{R}{2}}}^{\lambda + 1}
  }
  Y_{\lambda}^{0}\lr{\Omega}
\end{equation}
where $r_{<} = \min\lr{r, \tfrac{R}{2}}$, $r_{>} = \max\lr{r, \tfrac{R}{2}}$,
and where $E = \lrset{0, 2, 4, \dotsc}$ is the set of even integers.
Note that in computational implementations, we truncate this sum at some term,
$\lambda_{\max}$.
Whence, it follows that the matrix elements for this potential can be calculated
numerically to be of the form
\begin{alignat}{2}
  \label{eq:me-potential-axial}
  V_{i, j}
  =
  \bra*{\phi_{i}}
  \hat{V}
  \ket*{\phi_{j}}
  {}={}
  &
  -
  2
  \sum_{\lambda \in E}^{\lambda_{\max}}
  \lr[\bigg]
  {
    \int_{0}^{\infty}
    \dd{r}
    \varphi_{k_{i}, \ell_{i}}\lr{r}
    \dfrac
    {
      r_{<}^{\lambda}
    }
    {
      r_{>}^{\lambda + 1}
    }
    \varphi_{k_{j}, \ell_{j}}\lr{r}
  }
  \lr[\bigg]
  {
    \sqrt
    {
      \dfrac
      {
        4\pi
      }
      {
        2\lambda
        +
        1
      }
    }
    \bra*{Y_{\ell_{i}}^{m_{i}}}
    Y_{\lambda}^{0}\lr{\Omega}
    \ket*{Y_{\ell_{j}}^{m_{j}}}
  }
  \\
  &
  {}\times{}
  \delta_{\pi_{i}, \pi_{j}}
  \delta_{m_{i}, m_{j}}
  \nonumber
\end{alignat}
where $\pi_{i} = \lr{-1}^{\ell_{i}}$ is the parity quantum number.

\subsubsection*{Basis Symmetry}

Due to the axial symmetry of the $\mathrm{H}_{2}^{+}$ potential, and thus the
Hamiltonian for the electron in this molecule, we may choose a symmetrised
basis $\mathcal{B}^{\lr{m, \pi}}$ with specified azimuthal angular momentum,
$m$, and parity, $\pi$.
Furthermore, due to the computational constraints, we restrict our basis to
having $\lrset{N_{\ell}}_{\ell = 0}^{\ell_{\max}}$ radial functions per $\ell$,
with exponential falloffs $\lrset{\alpha_{\ell}}_{\ell = 0}^{\ell_{\max}}$.
For simplicity, we select $N_{\ell} = N_{0}$, and $\alpha_{\ell} = \alpha_{0}$,
for each $\ell = 0, \dotsc, \ell_{max}$.

\subsubsection*{Radial Grid}

We utilise a radial grid of the form
\begin{equation}
  \label{eq:radial-grid}
  \lrset
  {
    r_{i}
    =
    \mathrm{d_{r}}
    \cdot
    \lr{i - 1}
  }_{i = 1}^{\mathrm{n_{r}}}
\end{equation}
with $\mathrm{n_{r}}$ is the smallest integer such that
\begin{equation}
  \label{eq:radial-grid-n}
  \mathrm{d_{r}}
  \cdot
  \lr{\mathrm{n_{r}} - 1}
  \geq
  \mathrm{r}_{\max}
  .
\end{equation}
Note that when we numerically evaluate matrix elements of potentials which are
singular as $r \to 0$, we handle these integrals by searching for the first
$r_{i}$ for which $V_{i} = V\lr{r_{i}}$ is non-singular and evaluating the
integral from this point forwards.

\subsubsection*{Computational Parameters}

Across all computations performed, we have utilised $\alpha = 1$,
$\lambda_{\max} = 10$, $\mathrm{d_{r}} = 0.1$ and $\mathrm{r}_{\max} = 75$.
Hence, to denote a selection of computational parameters, we shall use the
notation
\begin{equation}
  \label{eq:basis-parameter}
  \mathcal{B}_{N_{0}, \ell_{\max}}^{\lr{m, \pi}}
\end{equation}
to indicate that the calculation has been performed with the Laguerre basis,
with $m$, $\pi$ symmetries, with $N_{0}$ radial functions per $\ell$, and with
$\ell = 0, \dotsc, \ell_{\max}$.

\subsubsection*{Computational Procedure}

The Hamiltonian for the $\mathrm{H}_{2}^{+}$ molecule was diagonalised in the
basis $\mathcal{B}_{N_{0}, \ell_{\max}}^{\lr{m, \pi}}$, for
$\pi \in \lrset{-1, +1}$, $m \in \lrset{0}$,
$N_{0} \in \lrset{2^{0}, \dotsc, 2^{5}}$, and
$\ell_{\max} \in \lrset{0, \dotsc, 6}$, across a range of axial-distance values
$R \in \lrset{0.5, 1.0, \dotsc, 10.0}$.

\subsection{Comparison with Accurate Potential-Energy Curve for $1s\sigma_{g}$.}

The convergence of the $\mathrm{H}_{2}^{+}$ potential-energy curve (PEC), for
the $1s\sigma_{g}$ state, is shown for increasing $\ell_{\max}$ in
\autoref{fig:pec-1ssg-lmax}, and for increasing $N_{0}$ in
\autoref{fig:pec-1ssg-n0}.
It can be seen by comparing the two figures that increasing $N_{0}$ past a
certain point does little to improve the accuracy of the PEC ($N_{0} = 2^{4}$
and $N_{0} = 2^{5}$ yield essentially identical PECs), while on the other hand,
increasing $\ell_{\max}$ continues to yield improved accuracy of the PEC (at
least up to $\ell_{\max} = 6$).

\begin{figure}[h]
  \centering
  \begin{tikzpicture}
    \begin{axis}[
      use units
      , scale = 1.8
      , title = {
        $\mathrm{H}_{2}^{+}$ PEC, for
        $\mathcal{B}_{32, \ell_{\max}}^{\lr{0, +}}$ with varying $\ell_{\max}$,
        compared with \lilf{PEC.1ssg}.
      }
      , grid = major
      , xmin = 0
      , xmax = 6
      , ymin = -0.65
      , ymax = 0.2
      , xlabel = {Axial Distance}
      , ylabel = {Energy}
      , x unit = {a_{0}}
      , y unit = {Ha}
      , legend entries
      , legend style = {
        cells = {anchor=east}
        , legend pos = north east
        , font = \tiny
      }
      ]

      \addplot [
      color = blue
      ] table [x index = 0, y index = 1]
      {../analytic_data/PEC.1ssg};
      \addlegendentryexpanded{$PEC.1s\sigma_{g}$}

      \newcommand{\pecfile}[3]{%
        ../extracted_data/pot/m-0.parity-#1.l_max-#2.n_basis_l_const-#3.txt}

      \pgfplotsforeachungrouped \l in {0, 2, 4, 6} {
        \pgfmathtruncatemacro{\i}{100.0*real(\l - 0)/real(6 - 0)}
        \edef\temp{
          \noexpand\addplot [
          color = red!\i!black
          ] table [col sep = comma, x index = 0, y index = 1]
          {\pecfile{1}{\l}{32}};
        }\temp
        \addlegendentryexpanded{$\ell_{\max} = \l$}
      }

    \end{axis}
  \end{tikzpicture}
  \caption
  {
    The potential energy curves, for $\ell_{\max} = 0, 2, 4, 6$, obtained by
    diagonalising the $\mathrm{H}_{2}^{+}$ Hamiltonian in the bases
    $\mathcal{B}_{32, \ell_{\max}}^{\lr{0, +}}$ (shown in black-to-red), are
    compared with the accuracte potential energy curve provided in
    \lilf{PEC.1ssg} (shown in blue).
    It can be seen that calculated PECs do converge to the accurate PEC, with a
    non-negligble increase in accuracy with increasing $\ell_{\max}$.
  }
  \label{fig:pec-1ssg-lmax}
\end{figure}
%
\begin{figure}[h]
  \centering
  \begin{tikzpicture}
    \begin{axis}[
      use units
      , scale = 1.8
      , title = {
        $\mathrm{H}_{2}^{+}$ PEC, for
        $\mathcal{B}_{N_{0}, 6}^{\lr{0, +}}$ with varying $N_{0}$,
        compared with \lilf{PEC.1ssg}.
      }
      , grid = major
      , xmin = 0
      , xmax = 6
      , ymin = -0.65
      , ymax = 0.2
      , xlabel = {Axial Distance}
      , ylabel = {Energy}
      , x unit = {a_{0}}
      , y unit = {Ha}
      , legend entries
      , legend style = {
        cells = {anchor=east}
        , legend pos = north east
        , font = \tiny
      }
      ]

      \addplot [
      color = blue
      ] table [x index = 0, y index = 1]
      {../analytic_data/PEC.1ssg};
      \addlegendentryexpanded{$PEC.1s\sigma_{g}$}

      \newcommand{\pecfile}[3]{%
        ../extracted_data/pot/m-0.parity-#1.l_max-#2.n_basis_l_const-#3.txt}

      \pgfplotsforeachungrouped \k in {0, ..., 5} {
        \pgfmathtruncatemacro{\n}{2^(\k)}
        \pgfmathtruncatemacro{\i}{100.0*real(\k - 0)/real(5 - 0)}
        \edef\temp{
          \noexpand\addplot [
          color = red!\i!black
          ] table [col sep = comma, x index = 0, y index = 1]
          {\pecfile{1}{6}{\n}};
        }\temp
        \addlegendentryexpanded{$N_{0} = 2^{\k}$}
      }

    \end{axis}
  \end{tikzpicture}
  \caption
  {
    The potential energy curves, for $N_{0} = 2^{0}, \dotsc, 2^{5}$, obtained by
    diagonalising the $\mathrm{H}_{2}^{+}$ Hamiltonian in the bases
    $\mathcal{B}_{N_{0}, 6}^{\lr{0, +}}$ (shown in black-to-red), are
    compared with the accuracte potential energy curve provided in
    \lilf{PEC.1ssg} (shown in blue).
    It can be seen that calculated PECs do converge to the accurate PEC, however
    the computational cost increases dramatically for diminishing improvements
    in the accuracy of the PEC.
  }
  \label{fig:pec-1ssg-n0}
\end{figure}

\clearpage

\subsection{Comparison with Accurate Potential-Energy Curve for $2p\sigma_{u}$.}

The convergence of the $\mathrm{H}_{2}^{+}$ potential-energy curve (PEC), for
the $2p\sigma_{u}$ state, is shown for increasing $\ell_{\max}$ in
\autoref{fig:pec-2psu-lmax}, and for increasing $N_{0}$ in
\autoref{fig:pec-2psu-n0}.
It can be seen by comparing the two figures that increasing $N_{0}$ past a
certain point does little to improve the accuracy of the PEC ($N_{0} = 2^{4}$
and $N_{0} = 2^{5}$ yield essentially identical PECs), while on the other hand,
increasing $\ell_{\max}$ continues to yield improved accuracy of the PEC (at
least up to $\ell_{\max} = 5$).

\begin{figure}[h]
  \centering
  \begin{tikzpicture}
    \begin{axis}[
      use units
      , scale = 1.8
      , title = {
        $\mathrm{H}_{2}^{+}$ PEC, for
        $\mathcal{B}_{32, \ell_{\max}}^{\lr{0, -}}$ with varying $\ell_{\max}$,
        compared with \lilf{PEC.2psu}.
      }
      , grid = major
      , xmin = 0.0
      , xmax = 6
      , ymin = -0.6
      , ymax = 1.0
      , restrict y to domain=-0.5:1.5
      , xlabel = {Axial Distance}
      , ylabel = {Energy}
      , x unit = {a_{0}}
      , y unit = {Ha}
      , legend entries
      , legend style = {
        cells = {anchor=east}
        , legend pos = north east
        , font = \tiny
      }
      ]

      \addplot [
      color = blue
      ] table [x index = 0, y index = 1]
      {../analytic_data/PEC.2psu};
      \addlegendentryexpanded{$PEC.2p\sigma_{u}$}

      \newcommand{\pecfile}[3]{%
        ../extracted_data/pot/m-0.parity-#1.l_max-#2.n_basis_l_const-#3.txt}

      \pgfplotsforeachungrouped \l in {1, 3, 5} {
        \pgfmathtruncatemacro{\i}{100.0*real(\l - 1)/real(6 - 1)}
        \edef\temp{
          \noexpand\addplot [
          color = red!\i!black
          ] table [col sep = comma, x index = 0, y index = 1]
          {\pecfile{-1}{\l}{32}};
        }\temp
        \addlegendentryexpanded{$\ell_{\max} = \l$}
      }

    \end{axis}
  \end{tikzpicture}
  \caption
  {
    The potential energy curves, for $\ell_{\max} = 1, 3, 5$, obtained by
    diagonalising the $\mathrm{H}_{2}^{+}$ Hamiltonian in the bases
    $\mathcal{B}_{32, \ell_{\max}}^{\lr{0, -}}$ (shown in black-to-red), are
    compared with the accuracte potential energy curve provided in
    \lilf{PEC.2psu} (shown in blue).
    It can be seen that calculated PECs do converge to the accurate PEC, with a
    non-negligble increase in accuracy with increasing $\ell_{\max}$.
  }
  \label{fig:pec-2psu-lmax}
\end{figure}
%
\begin{figure}[h]
  \centering
  \begin{tikzpicture}
    \begin{axis}[
      use units
      , scale = 1.8
      , title = {
        $\mathrm{H}_{2}^{+}$ PEC, for
        $\mathcal{B}_{N_{0}, 6}^{\lr{0, -}}$ with varying $N_{0}$,
        compared with \lilf{PEC.2psu}.
      }
      , grid = major
      , xmin = 0
      , xmax = 6
      , ymin = -0.6
      , ymax = 1.0
      , restrict y to domain=-0.5:1.5
      , xlabel = {Axial Distance}
      , ylabel = {Energy}
      , x unit = {a_{0}}
      , y unit = {Ha}
      , legend entries
      , legend style = {
        cells = {anchor=east}
        , legend pos = north east
        , font = \tiny
      }
      ]

      \addplot [
      color = blue
      ] table [x index = 0, y index = 1]
      {../analytic_data/PEC.2psu};
      \addlegendentryexpanded{$PEC.2p\sigma_{u}$}

      \newcommand{\pecfile}[3]{%
        ../extracted_data/pot/m-0.parity-#1.l_max-#2.n_basis_l_const-#3.txt}

      \pgfplotsforeachungrouped \k in {0, ..., 5} {
        \pgfmathtruncatemacro{\n}{2^(\k)}
        \pgfmathtruncatemacro{\i}{100.0*real(\k - 0)/real(5 - 0)}
        \edef\temp{
          \noexpand\addplot [
          color = red!\i!black
          ] table [col sep = comma, x index = 0, y index = 1]
          {\pecfile{-1}{6}{\n}};
        }\temp
        \addlegendentryexpanded{$N_{0} = 2^{\k}$}
      }

    \end{axis}
  \end{tikzpicture}
  \caption
  {
    The potential energy curves, for $N_{0} = 2^{0}, \dotsc, 2^{5}$, obtained by
    diagonalising the $\mathrm{H}_{2}^{+}$ Hamiltonian in the bases
    $\mathcal{B}_{N_{0}, 6}^{\lr{0, -}}$ (shown in black-to-red), are
    compared with the accuracte potential energy curve provided in
    \lilf{PEC.2psu} (shown in blue).
    It can be seen that calculated PECs do converge to the accurate PEC, however
    the computational cost increases dramatically for diminishing improvements
    in the accuracy of the PEC.
  }
  \label{fig:pec-2psu-n0}
\end{figure}

\clearpage

\section{$\mathrm{H}_{2}^{+}$ Vibrational Wave Functions}

\subsection*{Details of Relevant Theory and Code}

\subsection{Vibrational Wave Functions for $1s\sigma_{g}$ PEC.}

The vibrational wave-functions of $\mathrm{H}_{2}^{+}$, in the $1s\sigma_{g}$
state, were calculated by diagonalising vibrational Hamiltonian in a Laguerre
basis, $\mathcal{B}_{128, 0}^{\lr{0, +}}$, with the vibrational potential matrix
elements being calculated numerically, after interpolating \lilf{PEC.1ssg} onto
the radial grid.
The first 10 vibrational wavefunctions are shown in \autoref{fig:vib-1ssg}; note
that the wavefunctions have been shifted by their corresponding vibrational
energies, and scaled arbitrarily to make their presentation more convenient.

\begin{figure}[h]
  \centering
  \begin{tikzpicture}
    \begin{axis}[
      use units
      , scale = 1.8
      , title = {
        $\mathrm{H}_{2}^{+}$ Vibrational Wave Functions, for $1s\sigma_{g}$ PEC.
      }
      , grid = major
      , xmin = 0.0
      , xmax = 6.0
      , restrict x to domain=0.0:6.0
      , ymin = -0.61
      , ymax = -0.51
      , restrict y to domain=-1.0:-0.4
      , xlabel = {Axial Distance}
      , ylabel = {Ha}
      , x unit = {a_{0}}
      , y unit = {Energy}
      , legend entries
      , legend style = {
        cells = {anchor=east}
        , legend pos = south east
        , font = \tiny
      }
      ]

      \addplot [
      color = blue
      ] table [x index = 0, y index = 1]
      {../analytic_data/PEC.1ssg};
      \addlegendentryexpanded{$PEC.1s\sigma_{g}$}

      \newcommand{\vibfile}[2]{%
        ../extracted_data/vib/n_basis_l_const-#1.reduced_mass-#2.shifted_radial.txt}

      \pgfplotsforeachungrouped \k in {1, ..., 10} {
        \pgfmathtruncatemacro{\kk}{\k+1}
        \pgfmathtruncatemacro{\i}{100.0*real(\k - 1)/real(10 - 1)}
        \edef\temp{
          \noexpand\addplot [
          color = red!\i!black
          ] gnuplot [raw gnuplot] {
            plot "\vibfile{128}{918.07635}" every 2 u 1:\kk;
          };
        }\temp
        \addlegendentryexpanded{$\nu = \k$}
      }

    \end{axis}
  \end{tikzpicture}
  \caption
  {
    The first 10 vibrational wavefunctions, $\psi_{\nu}\lr{r}$, are shown in the
    form $a\psi_{\nu}\lr{r} - \epsilon_{\nu}$, where $a = 0.0035$, and
    $\epsilon_{\nu}$ is its corresponding vibrational energy.
    These wavefunctions were calculated by diagonalising the $H_{2}^{+}$
    vibrational Hamiltonian in a Laguerre basis
    $\mathcal{B}_{128, 0}^{\lr{0, +}}$, using the accurate $1s\sigma_{g}$ PEC
    (shown in blue).
  }
  \label{fig:vib-1ssg}
\end{figure}

\subsection{Lowest-Energy, $\nu = 0$, Vibrational Wave Functions for
  $1s\sigma_{g}$ PEC, for each Isotopologue of $\mathrm{H}_{2}^{+}$.}

The ground-state vibrational wavefunctions are shown for $\mathrm{H}_{2}^{+}$
and its isotopologues (indexed by their reduced mass, $\mu$) for the
$1s\sigma_{g}$ state in
\autoref{fig:vib-1ssg-isotopologue}.
Note that the wavefunctions have been shifted by their vibrational energy, and
scaled arbitrarily to allow for easier visual comparison.
It can be seen that the vibrational ground-state energy decreases as the reduced
mass of the isotopolgue increases.
However, the overall behaviour is very similar, being a uni-modal
wavefunction confined to the deepest region of the PEC.
As the reduced mass increases, the wavefunctions become increasingly confined to
this region.

\begin{figure}[h]
  \centering
  \begin{tikzpicture}
    \begin{axis}[
      use units
      , scale = 1.8
      , title = {
        $\mathrm{H}_{2}^{+}$ Isotopologue Vibrational Wave Functions, for
        $1s\sigma_{g}$ PEC.
      }
      , grid = major
      , xmin = 0.0
      , xmax = 6.0
      , restrict x to domain=0.0:6.0
      , ymin = -0.61
      , ymax = -0.59
      , restrict y to domain=-1.0:-0.4
      , xlabel = {Axial Distance}
      , ylabel = {Ha}
      , x unit = {a_{0}}
      , y unit = {Energy}
      , legend entries
      , legend style = {
        cells = {anchor=east}
        , legend pos = south east
        , font = \tiny
      }
      ]

      \addplot [
      color = blue
      ] table [x index = 0, y index = 1]
      {../analytic_data/PEC.1ssg};
      \addlegendentryexpanded{$PEC.1s\sigma_{g}$}

      \newcommand{\vibfile}[2]{%
        ../extracted_data/vib/n_basis_l_const-#1.reduced_mass-#2.shifted_radial.txt}

      \addplot [
      color = red!0!black
      ] gnuplot [raw gnuplot] {
        plot "\vibfile{128}{918.07635}" every 2 u 1:2;
      };
      \addlegendentryexpanded{$\mu = 918.07635$}

      \addplot [
      color = red!20!black
      ] gnuplot [raw gnuplot] {
        plot "\vibfile{128}{1223.89925}" every 2 u 1:2;
      };
      \addlegendentryexpanded{$\mu = 1223.89925$}

      \addplot [
      color = red!40!black
      ] gnuplot [raw gnuplot] {
        plot "\vibfile{128}{1376.39236}" every 2 u 1:2;
      };
      \addlegendentryexpanded{$\mu = 1376.39236$}

      \addplot [
      color = red!60!black
      ] gnuplot [raw gnuplot] {
        plot "\vibfile{128}{1835.24151}" every 2 u 1:2;
      };
      \addlegendentryexpanded{$\mu = 1835.24151$}

      \addplot [
      color = red!80!black
      ] gnuplot [raw gnuplot] {
        plot "\vibfile{128}{2200.87999}" every 2 u 1:2;
      };
      \addlegendentryexpanded{$\mu = 2200.87999$}

      \addplot [
      color = red!100!black
      ] gnuplot [raw gnuplot] {
        plot "\vibfile{128}{2748.46079}" every 2 u 1:2;
      };
      \addlegendentryexpanded{$\mu = 2748.46079$}

    \end{axis}
  \end{tikzpicture}
  \caption
  {
    The ground state, $\nu = 0$, vibrational wavefunctions for
    $\mathrm{H}_{2}^{+}$ and its isotopologues, are shown in the form
    $a\psi_{0}^{\lr{\mu}}\lr{r} - \epsilon_{\nu}^{\lr{\mu}}$, where
    $a = 0.0035$, and $\epsilon_{\nu}$ is the vibrational energy corresponding
    to the isotopologues ground state.
    These wavefunctions were calculated by diagonalising the $H_{2}^{+}$
    vibrational Hamiltonian in a Laguerre basis
    $\mathcal{B}_{128, 0}^{\lr{0, +}}$, using the accurate $1s\sigma_{g}$ PEC
    (shown in blue).
    Note that the phase-shift inversion of the first two isotopologues is a
    by-product of the diagonalisation procedure and has no physical
    significance.
  }
  \label{fig:vib-1ssg-isotopologue}
\end{figure}

\end{document}